% --- Collaborative Filtering ---
\begin{tikzpicture}
    \node [mybox] (box){%
        \begin{minipage}{0.48\textwidth}
        \textbf{User-based CF} aims to find similar users and recommend items based on their preferences.\par
        \begin{tabular}{lp{0.5\textwidth} l}
            Similarity & Formula & Context \\
            \hline
            Cosine Similarity & $s_{ij} = \frac{\vec{i} \cdot \vec{j}}{\|\vec{i}\| \cdot \|\vec{j}\|}$ & Implicit ratings \\
            Jaccard Similarity & $s_{ij} = \frac{\vec{i} \cap \vec{j}}{\vec{i} \cup \vec{j}}$ & Implicit ratings \\
            Pearson Correlation & 
            $s_{ij} = 
            \frac{
                \sum_{i \in I} 
                    (r_{iu} - \bar{r}_u) \cdot
                    (r_{iv} - \bar{r}_v)
                }
                {
                    \sqrt{\sum_{i \in I} (r_{iu} - \bar{r}_u)^2} 
                    \sqrt{\sum_{i \in I} (r_{iv} - \bar{r}_v)^2}
                }$ & Explicit ratings \\
        \end{tabular}
        \begin{tcolorbox}[colback=gray!5, colframe=gray!80!black, title={Focus on Pearson Correlation}]
            Pearson correlation computes similarity between a rating delta. Therefore the similarity is used to predict the delta of the rating!
            \\
            \begin{center}
            $\tilde{r}_{ui} - \bar{r}_u = 
               \frac{
                \sum{v \in KNN(u)}(r_{vi} - \bar{r}_{v})\cdot s_{uv}
               }
               {
                \sum{v \in KNN(u)} s_{uv}
               }
            $
            \end{center}
            

            
        \end{tcolorbox}

        \begin{tcolorbox}[colback=gray!5, colframe=gray!80!black, title={The Top-N Scenario}]
            Normalizing the predicted rating is necessary to improve rating prediction.

            In a Top-N scenario, we can compute the rating $\tilde{r}_{ui}$ without normalization to save computation!
        \end{tcolorbox}

        \textbf{Item-based CF} aims to find similarity between items based how many users have the same opinion about them. The similarity is obtained in the same way as for user-based CF, considering the items instead of the users.\par

        \begin{tcolorbox}[colback=gray!5, colframe=gray!80!black, title={Memory-Based vs Model-Based}]
            \textbf{Memory-Based:}
            \begin{itemize}[label={--}, topsep=0cm, parsep=0cm, itemsep=0cm]
                \item Requires user profile in URM used to build model
                \item Only works for "known" users
                \item Must rebuild model for new users
                \item Example: User-Based CF (uses user neighborhood)
            \end{itemize}
            
            \textbf{Model-Based:}
            \begin{itemize}[label={--}, topsep=0cm, parsep=0cm, itemsep=0cm]
                \item Works with any user profile
                \item Supports both "known" and "unknown" users
                \item No model recomputation needed for new users
                \item Example: Item-Based CF (uses item similarities)
            \end{itemize}
        \end{tcolorbox}

        \begin{tcolorbox}[colback=gray!5, colframe=gray!80!black, title={Association Rules}]
            Association rules explore relationships between items using conditional probability:
            \begin{center}
            $P(i|j) = \frac{\text{\# appearances of i and j}}{\text{\# appearances of j} + C}$
            \end{center}
            where C is a shrinkage term to avoid biases. The similarity is asymmetric: $P(i|j) \neq P(j|i)$
        \end{tcolorbox}

        
        \end{minipage}
    };

% --- Collaborative Filtering header ---
\node[fancytitle, right=10pt] at (box.north west) {Collaborative Filtering};

\end{tikzpicture}