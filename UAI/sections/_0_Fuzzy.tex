\begin{tikzpicture}
    \node[mybox] (box) {
        \begin{minipage}{0.48\textwidth}
            \begin{tabular}{lp{0.65\textwidth} l}
                Fuzzy Set &
                A set whos membership function range on interval $[0,1]$
                \\
                Membership Function &
                Defins a set by defining the degree of membership of an element of the universe of discourse.
                \\
                Frame of Cognition &
                A set of fuzzy sets fully covering the universe of discourse(the range of variables)
                \begin{itemize}[label={--}, topsep=0cm, parsep=0cm, itemsep=0cm]
                    \item Coverage
                    \item Unimodality
                \end{itemize}
                \\
                Fuzzy Partition &
                A frame of cognition for which the sum of the membership values of each value of the base variable is equal to 1.
                \\
                $\alpha$-Cut &
                Is the crisp set of the values of $x$ such that $\mu(X) \geq \alpha$
                \begin{center}
                    $\alpha \text{-cut}_\mu(X) = \{\mu(x) \geq \alpha\}$
                \end{center}
                \\
                Support &
                The crisp set of values $x \in X$ such that $\mu(x) > 0$
                \\
                Height of a Fuzzy Set &
                The height of a fuzzy set $f$ is the highest membership degree of an element $x$ of the fuzzy set.
                \begin{center}
                    $h_f(X) = \max_{x \in X} \mu_f(x)$
                \end{center}
                \\
                Normal Fuzzy Set &
                A fuzzy set is normal $ \iff  h_f(X) = 1$
                \\
                Convex Fuzzy Set &
                A fuzzy set is convex $ \iff \forall x_1, x_2 \in X, \forall \lambda \in [0,1]$
                \begin{center}
                    $\mu_f(\lambda x_1 + (1-\lambda)x_2) \geq \min(\mu_f(x_1), \mu_f(x_2))$
                \end{center}
            \end{tabular}
        \end{minipage}
    };

    \node[fancytitle, right=10pt] at (box.north west) {Fuzzy Sets};
\end{tikzpicture}


\begin{tikzpicture}
    \node[mybox] (box) {
        \begin{minipage}{0.48\textwidth}
            \begin{tabular}{lp{0.65\textwidth} l}
                Fuzzy Logic &
                Infinite-valued logic, with truth values is [0..1]
                $
                    A \text{ is } L
                $
                Where:
                \begin{itemize}[label={--}, topsep=0cm, parsep=0cm, itemsep=0cm]
                    \item $A$ is a linguistic variable
                    \item $L$ is a label of a fuzzy set
                \end{itemize}
                \\
                Linguistic Variable &
                A is $(X,T(X),U,G,M)$
                Where:
                \begin{itemize}[label={--}, topsep=0cm, parsep=0cm, itemsep=0cm]
                    \item $X$ is the universe of discourse
                    \item $T(X)$ is the set of linguistic terms
                    \item $U$ is the set of values of the linguistic variable
                    \item $G$ is the semantic rule
                    \item $M$ is the mapping of the semantic rule
                \end{itemize}
                \\
                Fuzzy Modifiers &
                A fuzzy modifier is a linguistic variable that is used to modify the meaning of a linguistic variable.
                \begin{itemize}[label={--}, topsep=0cm, parsep=0cm, itemsep=0cm]
                    \item Strong Modifier
                            - $m(a) \leq a \forall a \in [0 \ldots 1]$

                            - Prediction stronger, decrease the truth value
                    \item Weak Modifier
                            - $m(a) \geq a \forall a \in [0 \ldots 1]$

                            - Prediction weaker, increase the truth value
                \end{itemize}
                \\
                Modifiers Properties &
                \begin{itemize}[label={--}, topsep=0cm, parsep=0cm, itemsep=0cm]
                    \item $m(0) = 0$ and $m(1) = 1$
                    \item $m$ is a continuous function
                    \item if $m$ is strong, $m^{-1}$ is weak and vice versa
                    \item a composition of modifiers is a modifier. If both are of the same type, the result is of the same type
                \end{itemize}
                

                
            \end{tabular}
        \end{minipage}
    };

    \node[fancytitle, right=10pt] at (box.north west) {Fuzzy Logic};
\end{tikzpicture}